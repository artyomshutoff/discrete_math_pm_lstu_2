\documentclass{article}
\usepackage[utf8]{inputenc}
\usepackage [warn] {mathtext}
\usepackage{graphicx}
\usepackage[english, russian]{babel}
\setlength{\parindent}{0pt}
\usepackage[table,xcdraw]{xcolor}

\title{Дискретная математика. Лекция 25.02.}
\author{С. В. Ткаченко}
\date{25.02.2022}

\begin{document}
	\maketitle
	\textbf{Высказывание} - это предложение, смысл которого может быть
	\textbf{истинным} или \textbf{ложным}.
	\\\\
	Если суждение, составляющее смысл некоторого высказывания, истинно,
	то высказывание истинно.
	\\\\
	Если суждение, составляющее смысл некоторого высказывания, ложно,
	то высказывание ложно.
	\\\\
	Истинность и ложность называются логическими, или истинностными,
	значениями высказываний.
	\\\\
	\textbf{Сложное высказывание} - это высказывание, составленное из
	других высказываний с помощью логических операций.
	\\\\\
	\textbf{Элементарное высказывание} - это высказывание, которое
	представляет собой только одно утверждение. Такие высказывания
	утверждают что-то о \textit{свойствах} объекта или об \textit{отношениях}
	между объектами (чаще всего - двумя).
	\\\\
	Обозначение высказываний: A, B, C, ...
	\\\\
	значения высказываний: Л - ложь, И - истина.
	\\\\
	
	\begin{center}
		\subsection*{Операции над высказываниями}
	\end{center}
	
	Пусть даны два произвольных высказывания A и B.
	\\\\
	1. \textbf{Отрицанием} высказывания A называется высказывание,
	истинное тогда и только тогда, когда высказывание A ложно.
	\\\\
	Обозначается $\bar{A}$ (или $\neg A$, $A'$), читается "не A".	
	\\\\
	2. \textbf{Конъюнкцией} двух высказываний A и B называется
	высказывание, истинное тогда и только тогда, когда оба
	высказывания истинны.
	\\\\
	Обозначается $A \wedge B$ (или $A \& B$), читается "A и B".
	\\\\
	3. \textbf{Дизъюнкцией} двух высказываний A и B называется
	высказывание, ложное тогда и только тогда, когда оба
	высказывания ложны.
	\\\\
	Обозначается $A \vee B$, читается "A или B".
	\\\\
	4. \textbf{Импликацией} двух высказываний A и B называется
	высказывание, ложное тогда и только тогда, когда A истинно,
	а B ложно.
	\\\\
	Обозначается $A \rightarrow B$  (или $A \supset B$, $A \Rightarrow B$),
	читается "A влечет B" (или "если A, то B"$,$ "из A следует B").
	\\\\
	5. \textbf{Эквивалентностью} двух высказываний A и B называется
	высказывание, истинное тогда и только тогда, когда истинностные
	значения A и B совпадают.
	\\\\
	Обозначается $A \sim B$, читается "A эквивалентно B".
	\\\\
	6. \textbf{Суммой по mod 2} двух высказываний A и B называется высказывание,
	истинное тогда и только тогда, когда истинностные значения A и B различны.
	\\\\
	Обозначается $A \oplus B$, читается "A сумма по модулю 2 B".
	\\\\
	7. \textbf{Штрих Шеффера} - антиконъюнкция.\\
	\textbf{Антиконъюнкцией} двух высказываний A и B называется высказывание,
	ложное тогда и только тогда, когда оба высказывания истинны.
	\\\\
	Обозначается $(A | B) = \overline{(A \wedge B)}$, читается "A штрих Шеффера B".
	\\\\
	8. \textbf{Стрелка Пирса} - антидизъюнкция.\\
	\textbf{Антидизъюнкцией} двух высказываний A и B называется высказывание,
	истинное тогда и только тогда, когда оба высказывания ложны.
	\\\\
	Обозначается $(A \downarrow B) = \overline{(A \vee B)}$, читается "A стрелка пирса B".
	
	\begin{center}
		\item
		\subsection*{Булевы функции. Представления булевой функции формулой алгебры высказываний}
	\end{center}
	
	\textbf{Булевой функцией} $f (x_1, ..., x_n)$ называется произвольная n-местная функция,
	действующая из множества ${\{0, 1\}}^n$ во множество ${\{0, 1\}}$:
	\\\\
	$f (x_1, ..., x_n): {\{0, 1\}}^n \rightarrow {\{0, 1\}}$\\
	аргументы функции $x_1, ..., x_n$ принимают значения 0 или 1,\\
	функция f также принимает значения 0 или 1.
	\\\\
	Пусть значению Л соответствует значение 0,\\
	значению И соответствует значение 1.\\
	\\\\
	Тогда каждой формуле алгебры высказываний $F$ можно поставить в соответствие
	булеву функцию $f$.
	\\\\
	Представление булевой функции таблицей истинности.

	\begin{table}[ht]
		\begin{tabular}{|l|l|l|l|l|}
			\hline
			\rowcolor[HTML]{D7B1DA} 
			$x_1$ & $x_2$ & $x_3$ & Число & $f(x_1, x_2, x_3)$ \\ \hline
			\rowcolor[HTML]{C8D7A5} 
			0     & 0     & 0     & 0     & f(0, 0, 0)         \\ \hline
			\rowcolor[HTML]{C8D7A5} 
			0     & 0     & 1     & 1     & f(0, 0, 1)         \\ \hline
			\rowcolor[HTML]{C8D7A5} 
			0     & 1     & 0     & 2     & f(0, 1, 0)         \\ \hline
			\rowcolor[HTML]{C8D7A5} 
			0     & 1     & 1     & 3     & f(0, 1, 1)         \\ \hline
			\rowcolor[HTML]{BBDAD5} 
			1     & 0     & 0     & 4     & f(1, 0, 0)         \\ \hline
			\rowcolor[HTML]{BBDAD5} 
			1     & 0     & 1     & 5     & f(1, 0, 1)         \\ \hline
			\rowcolor[HTML]{BBDAD5} 
			1     & 1     & 0     & 6     & f(1, 1, 0)         \\ \hline
			\rowcolor[HTML]{BBDAD5} 
			1     & 1     & 1     & 7     & f(1, 1, 1)         \\ \hline
		\end{tabular}
	\end{table}
	
	Каждая строка таблицы - двоичная запись числа из множества $\{0, 1, 2, ..., 2^n - 1\}$
	\\\\
	\underline{\textbf{Пример.}} $f (x_1, x_2, x_3) = \bar{x_3} \rightarrow (x_1 \sim x_2)$\\

	\begin{table}[ht]
		\begin{tabular}{|l|l|l|l|l|l|}
			\hline
			\rowcolor[HTML]{E2C8E4} 
			$x_1$ & $x_2$ & $x_3$ & $\bar{x_3}$ & $x_1 \sim x_2$ & $f(x_1, x_2, x_3)$ \\ \hline
			\rowcolor[HTML]{D5D7A5} 
			0     & 0     & 0     & 1           & 1              & 1                  \\ \hline
			\rowcolor[HTML]{D5D7A5} 
			0     & 0     & 1     & 0           & 1              & 1                  \\ \hline
			\rowcolor[HTML]{D5D7A5} 
			0     & 1     & 0     & 1           & 0              & 0                  \\ \hline
			\rowcolor[HTML]{D5D7A5} 
			0     & 1     & 1     & 0           & 0              & 1                  \\ \hline
			\rowcolor[HTML]{C9E4C5} 
			1     & 0     & 0     & 1           & 0              & 0                  \\ \hline
			\rowcolor[HTML]{C9E4C5} 
			1     & 0     & 1     & 0           & 0              & 1                  \\ \hline
			\rowcolor[HTML]{C9E4C5} 
			1     & 1     & 0     & 1           & 1              & 1                  \\ \hline
			\rowcolor[HTML]{C9E4C5} 
			1     & 1     & 1     & 0           & 1              & 1                  \\ \hline
		\end{tabular}
	\end{table}

	Если булева функция $f$ зависит от $n$ переменных $(x_1, ..., x_n)$, то существует ровно $2^{2^n}$
	различных n-местных булевых функций.
	\\\\
	При $n = 1:$ \qquad $2^{2^n} = 2^{2^1} = 4$ функции.
	\\\\
	При $n = 2:$ \qquad $2^{2^n} = 2^{2^2} = 16$ функций.
	\\\\
	При $n = 3:$ \qquad $2^{2^n} = 2^{2^3} = 256$ функций.
	\\\\
	Булева функция $f (x_1, ...,$ $x_{i-1},$ $x_i,$ $x_{i+1},$ $...$, $ x_n)$
	\textit{существенно зависит} от переменной $x_i$, если существует такой набор значений
	$\alpha_1$, $...$, $\alpha_{i-1}$, $\alpha_{i+1}$, $...$, $\alpha_n$, что
	\begin{center}
	$f(\alpha_1$, $...$, $\alpha_{i-1}$, $0$, $\alpha_{i+1}$, $...$, $\alpha_n)$ $\neq$
	$f(\alpha_1$, $...$, $\alpha_{i-1}$, $1$, $\alpha_{i+1}$, $...$, $\alpha_n).$
	\end{center}

	В этом случае $x_i$ называют \textit{существенной} переменной, в противном случае $x_i$
	называют \textit{фиктивной} переменной.

	\begin{center}
		\subsection*{Булевы функции одной переменной}
	\end{center}

	\begin{table}[ht]
		\begin{tabular}{|l|l|l|l|l|}
			\hline
			\rowcolor[HTML]{E2C8E4} 
			& Переменная x    & 0 & 1 &           \\ \hline
			\rowcolor[HTML]{9FC0AF} 
			Название          & Обозначение     &   &   & Фиктивная \\ \hline
			\rowcolor[HTML]{CFB9A9} 
			константа ноль    & $f_0 = 0$       & 0 & 0 & x         \\ \hline
			\rowcolor[HTML]{BCAC8E} 
			тождественная x   & $f_1 = x$       & 0 & 1 &           \\ \hline
			\rowcolor[HTML]{A5D9C8} 
			отрицание x       & $f_2 = \bar{x}$ & 1 & 0 &           \\ \hline
			\rowcolor[HTML]{B8BCE4} 
			константа единица & $f_3 = 1$       & 1 & 1 & x         \\ \hline
		\end{tabular}
	\end{table}

	\begin{center}
	\subsection*{Булевы функции двух переменных}
	\end{center}

	\begin{table}[ht]
		\begin{tabular}{|l|l|l|l|l|l|l|}
			\hline
			\rowcolor[HTML]{B0B2D9} 
			& Переменная x                         & 0 & 0 & 1 & 1 &           \\ \hline
			\rowcolor[HTML]{A9CAB3} 
			Название          & Обозначение                          &   &   &   &   & Фиктивные \\ \hline
			\rowcolor[HTML]{C8A5B5} 
			константа ноль    & $f_0 = 0$                            & 0 & 0 & 0 & 0 & x, y      \\ \hline
			\rowcolor[HTML]{C8A5B5} 
			конъюнкция        & $f_1 = x \wedge y$                   & 0 & 0 & 0 & 1 &           \\ \hline
			\rowcolor[HTML]{C8A5B5} 
			запрет по y       & $f_2 = \overline{(x \rightarrow y)}$ & 0 & 0 & 1 & 0 &           \\ \hline
			\rowcolor[HTML]{C8A5B5} 
			тождественная x   & $f_3 = x$                            & 0 & 0 & 1 & 1 & y         \\ \hline
			\rowcolor[HTML]{CAA788} 
			запрет по x       & $f_4 \overline{(y \rightarrow x)}$   & 0 & 1 & 0 & 0 &           \\ \hline
			\rowcolor[HTML]{CAA788} 
			тождественная y   & $f_5 = y$                            & 0 & 1 & 0 & 1 & x         \\ \hline
			\rowcolor[HTML]{CAA788} 
			сумма по mod 2    & $f_6 = x \oplus y$                   & 0 & 1 & 1 & 0 &           \\ \hline
			\rowcolor[HTML]{CAA788} 
			дизъюнкция        & $f_7 = x \vee y$                     & 0 & 1 & 1 & 1 &           \\ \hline
			\rowcolor[HTML]{9DCBAB} 
			стрелка Пирса     & $f_8 = x \downarrow y$               & 1 & 0 & 0 & 0 &           \\ \hline
			\rowcolor[HTML]{9DCBAB} 
			эквивалентность   & $f_9 = x \sim y$                     & 1 & 0 & 0 & 1 &           \\ \hline
			\rowcolor[HTML]{9DCBAB} 
			отрицание y       & $f_{10} = \bar{y}$                   & 1 & 0 & 1 & 0 &           \\ \hline
			\rowcolor[HTML]{9DCBAB} 
			конверсия         & $f_{11} = y \rightarrow x$           & 1 & 0 & 1 & 1 &           \\ \hline
			\rowcolor[HTML]{B9B0D0} 
			отрицание x       & $f_{12} = \bar{x}$                   & 1 & 1 & 0 & 0 & y         \\ \hline
			\rowcolor[HTML]{B9B0D0} 
			импликация        & $f_{13} = x \rightarrow y$           & 1 & 1 & 0 & 1 &           \\ \hline
			\rowcolor[HTML]{B9B0D0} 
			штрих Шеффера     & $f_{14} = x | y$                     & 1 & 1 & 1 & 0 &           \\ \hline
			\rowcolor[HTML]{B9B0D0} 
			константа единица & $f_{15} = 1$                         & 1 & 1 & 1 & 1 & x, y      \\ \hline
		\end{tabular}
	\end{table}
	
	\textit{Формула алгебры высказываний} - это сложное высказывание, составленное из
	элементарных высказываний с помощью операций 1 - 8.
	\\\\
	\underline{\textbf{Пример.}}\\
	Пусть высказывание X принимает значение Л,
	высказывание Y - Л,\\
	высказывание Z - И,
	\\\\
	тогда формула
	\begin{center}
		$A = (Y \wedge (Z \rightarrow X)) \vee (\bar{X} \sim \bar{Y})$
	\end{center}
	примет значение
	\begin{center}
		$A = (Л \wedge (И \rightarrow Л)) \vee (И \sim И) = (Л \wedge Л) \vee И = Л \vee И = И$.
	\end{center}
	
\end{document}