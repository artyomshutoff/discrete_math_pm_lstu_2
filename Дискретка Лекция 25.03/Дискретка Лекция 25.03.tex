\documentclass{article}
\usepackage[utf8]{inputenc}
\usepackage [warn] {mathtext}
\usepackage{graphicx}
\usepackage[english, russian]{babel}
\setlength{\parindent}{0pt}
\usepackage[table,xcdraw]{xcolor}
\usepackage{booktabs}

\title{Дискретная математика. Лекция 25.03.}
\author{С. В. Ткаченко}
\date{25.03.2022}

\begin{document}
	\maketitle
	\textit{Алгеброй Жегалкина} называют алгебру на множестве булевых функций,
	которая включает две операции: конъюнкцию ($\wedge$) и сумму по mod 2 ($\oplus$),
	а также константы 0 и 1.\\
	
	\begin{center}
	\subsection*{\textbf{Равносильности алгебры Жегалкина}}
	\end{center}
	
	\begin{table}[ht]
		\centering
		\begin{tabular}{|l|l|} 
			\hline
			\multicolumn{2}{|l|}{\textbf{Коммутативность}} \\ 
			\hline
			$x \wedge y = y \wedge x$ & $x \oplus y = y \oplus x$ \\ 
			\hline
			\multicolumn{2}{|l|}{\textbf{Ассоциативность}} \\ 
			\hline
			$x \wedge (y \wedge z) = (x \wedge y) \wedge z$ & $x \oplus (y \oplus z) = (x \oplus y) \oplus z$ \\ 
			\hline
			\multicolumn{2}{|l|}{\textbf{Дистрибутивность}} \\ 
			\hline
			\multicolumn{2}{|l|}{$x \wedge (y \oplus z) = (x \wedge y) \oplus (x \wedge z)$} \\ 
			\hline
			\multicolumn{2}{|l|}{\begin{tabular}[c]{@{}l@{}}\textbf{Равносильности идемпотентности, дополнения,}\\\textbf{тождества и констант}\end{tabular}} \\ 
			\hline
			$x \wedge x = x$ & $x \oplus x = 0$ \\ 
			\hline
			$x \wedge 0 = 0$ & $x \oplus 0 = x$ \\ 
			\hline
			$x \wedge 1 = x$ & $x \oplus 1 = \overline{x}$ \\
			\hline
		\end{tabular}
	\end{table}
	
	\subsection*{\textbf{Замена операций}}
	
	$1) x \vee y = \overline{(\overline{x} \wedge \overline{y})} = (x \oplus 1) \wedge (y \oplus 1) \oplus 1 = x \wedge y \oplus x \oplus y;$
	\\\\
	$2) x \sim y = \overline{(\overline{x} \oplus \overline{y})} = \overline{x} \oplus \overline{y} \oplus 1 = x \oplus 1 \oplus y \oplus 1 \oplus 1 = x \oplus y \oplus 1.$
	\\\\
	\textit{Полином Жегалкина} функции $f(x_1, ..., x_n)$ называется полином вида\\
	\begin{center}
	$P(x_1, ..., x_n) = $
	\\
	$a_0 \oplus \sum\limits_{i=1}^{n} a_i \wedge x_i \oplus \sum\limits_{i, j = 1; i \neq j}^{n} a_{ij} \wedge x_i \wedge x_j \oplus ... \oplus$
	\\
	$a_{12...n} \wedge x_1 \wedge x_2 \wedge ... \wedge x_n,$
	\end{center}
	где коэффициенты $a_0, a_i, a_{ij}, ..., a_{12...n}$ принимают значение 0 или 1.
	\\
	\textbf{Теорема 9 (теорема Жегалкина).} Каждая булева функция $f(x_1, ..., x_n)$ может быть представлена в виде полинома Жегалкина и притом
	единственным образом, с точностью до порядка слагаемых.
	\\\\
	\underline{\textit{\textbf{Пример.}}} Построить полином Жегалкина для
	\begin{center}
		$f(x, y, z) = (\overline{y} \sim x) \vee \overline{z}.$
	\end{center}

	\textbf{1 способ}\\
	$f (x, y, z) = (\overline{y} \sim x) \vee \overline{z} =$\\
	$= [(\overline{y} \sim x) \wedge \overline{z}] \oplus [\overline{y} \sim x] \oplus [\overline{z}] =$\\
	$= [(\overline{y} \oplus x \oplus 1) \wedge (z \oplus 1)] \oplus [\overline{y} \oplus x \oplus 1] \oplus [z \oplus 1] =$\\
	$= [(y \oplus 1 \oplus x \oplus 1) \wedge (z \oplus 1)] \oplus [y \oplus 1 \oplus x \oplus 1] \oplus [z \oplus 1] =$\\
	$= [(y \oplus x) \wedge (z \oplus 1)] \oplus [y \oplus x] \oplus [z \oplus 1] =$\\
	$= [yz \oplus y \oplus xz \oplus x] \oplus [y \oplus x] \oplus [z \oplus 1] =$\\
	$= yz \oplus y \oplus xz \oplus x \oplus y \oplus x \oplus z \oplus 1 = (y \oplus y = 0, x \oplus x = 0) =$\\
	$= 1 \oplus z \oplus xz \oplus yz = P(x, y, z).$
	\\\\
	\textbf{2 способ}\\
	$P(x, y, z) = a_0 \oplus a_1 x \oplus a_2 y \oplus a_3 z \oplus a_{12} xy \oplus a_{13} xz \oplus a_{23} yz \oplus a_{123}xyz$
	
	\begin{table}[ht]
		\begin{tabular}{|l|l|l|l|l|l|l|l|}
			\hline
			\rowcolor[HTML]{D1A8CE} 
			x & y & z & $\overline{y}$ & $\overline{y} \sim x$ & $\overline{z}$ & $f(x, y, z)$ & Коэффициент \\ \hline
			\rowcolor[HTML]{CFD7A6} 
			0 & 0 & 0 &                &                       &                & 1            & $a_0$       \\ \hline
			\rowcolor[HTML]{CFD7A6} 
			0 & 0 & 1 &                &                       &                & 0            & $a_3$       \\ \hline
			\rowcolor[HTML]{CFD7A6} 
			0 & 1 & 0 &                &                       &                & 1            & $a_2$       \\ \hline
			\rowcolor[HTML]{CFD7A6} 
			0 & 1 & 1 &                &                       &                & 1            & $a_{23}$    \\ \hline
			\rowcolor[HTML]{B6D4C7} 
			\begin{tabular}[c]{@{}l@{}}1\\ 1\\ 1\end{tabular} &
			\begin{tabular}[c]{@{}l@{}}0\\ 0\\ 1\end{tabular} &
			\begin{tabular}[c]{@{}l@{}}0\\ 1\\ 0\end{tabular} &
			&
			&
			&
			\begin{tabular}[c]{@{}l@{}}1\\ 1\\ 1\end{tabular} &
			\begin{tabular}[c]{@{}l@{}}$a_1$\\ $a_{13}$\\ $a_{12}$\end{tabular} \\ \hline
			\rowcolor[HTML]{B6D4C7} 
			1 & 1 & 1 &                &                       &                & 0            & $a_{123}$   \\ \hline
		\end{tabular}
	\end{table}
	
	\begin{table}[ht]
		\begin{tabular}{|l|l|l|}
			\hline
			\rowcolor[HTML]{D1A8CE} 
			x & y & $x \oplus y$ \\ \hline
			\rowcolor[HTML]{CFD7A6} 
			0 & 0 & 0            \\ \hline
			\rowcolor[HTML]{CFD7A6} 
			0 & 1 & 1            \\ \hline
			\rowcolor[HTML]{B6D4C7} 
			1 & 0 & 1            \\ \hline
			\rowcolor[HTML]{B6D4C7} 
			1 & 1 & 0            \\ \hline
		\end{tabular}
	\end{table}
	
	$P(0, 0, 0) = a_0 = 1 \Rightarrow a_0 = 1,$\\
	$P(0, 0, 1) = a_0 \oplus a_3 = 0, \; 1 \oplus a_3 = 0 \Rightarrow a_3 = 1,$\\
	$P(0, 1, 0) = a_0 \oplus a_2 = 1, \; 1 \oplus a_2 = 1 \Rightarrow a_2 = 0,$\\
	$P(0, 1, 1) = a_0 \oplus a_2 \oplus a_3 \oplus a_{23} = 1, 1 \oplus 0 \oplus 1 \oplus a_{23} = 1 \Rightarrow a_{23} = 1,$\\
	$P(1, 0, 0) = a_0 \oplus a_1 = 1, \; 1 \oplus a_1 = 1 \Rightarrow a_1 = 0,$\\
	$P(1, 0, 1) = a_0 \oplus a_1 \oplus a_3 \oplus a_{13} = 1, 1 \oplus 0 \oplus 1 \oplus a_{13} = 1 \Rightarrow a_{13} = 1,$\\
	$P(1, 1, 0) = a_0 \oplus a_1 \oplus a_2 \oplus a_{12} = 1, 1 \oplus 0 \oplus 0 \oplus a_{12} = 1 \Rightarrow a_{12} = 0,$\\
	$P(1, 1, 1) = a_0 \oplus a_1 \oplus a_2 \oplus a_3 \oplus a_{12} \oplus a_{13} \oplus a_{23} \oplus a_{123} =$\\
	$1 \oplus 0 \oplus 0 \oplus 1 \oplus 0 \oplus 1 \oplus 1 \oplus a_{123} = 0,$
	
\end{document}