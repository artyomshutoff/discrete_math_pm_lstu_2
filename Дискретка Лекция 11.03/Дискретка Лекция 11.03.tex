\documentclass{article}
\usepackage[utf8]{inputenc}
\usepackage [warn] {mathtext}
\usepackage{graphicx}
\usepackage[english, russian]{babel}
\setlength{\parindent}{0pt}
\usepackage[table,xcdraw]{xcolor}
\usepackage{booktabs}

\title{Дискретная математика. Лекция 11.03.}
\author{С. В. Ткаченко}
\date{11.03.2022}

\begin{document}
	\maketitle
	
	Две формулы $A$ и $B$ называются \textit{равносильными}, если они
	принимают одинаковые значения на одном и том же списке переменных
	$X_1, X_2, ..., X_n,$ входящих в $A$ и $B$.\\
	Обозначение $A \equiv B$.
	
	\begin{center}
		\subsection*{Основные равносильности формул}
	
	Для любых формул $A, B, C$ справедливы следующие
	равносильности
	\end{center}
	
	\begin{table}[ht]
		\centering
		\begin{tabular}{|l|l|}
			\hline
			\begin{tabular}[c]{@{}l@{}}1. Коммутативность\\ а) $A \vee B \equiv B \vee A$,\\ б) $A \wedge B \equiv B \wedge A$.\end{tabular} &
			\begin{tabular}[c]{@{}l@{}}2. Ассоциативность\\ а) $A \vee (B \vee C) \equiv (A \vee B) \vee C$,\\ б) $A \wedge (B \wedge C) \equiv (A \wedge B) \wedge C$.\end{tabular} \\ \hline
			\begin{tabular}[c]{@{}l@{}}3. Дистрибутивность\\ а) $A \vee (B \wedge C) \equiv (A \vee B) \wedge (A \vee C)$,\\ б) $A \wedge (B \vee C) \equiv (A \wedge B) \vee (A \wedge C)$.\end{tabular} &
			\begin{tabular}[c]{@{}l@{}}4. Равносильность де Моргана\\ а) $\overline{(A \vee B)} \equiv \overline{A} \wedge \overline{B}$,\\ б) $\overline{(A \wedge B)} \equiv \overline{A} \vee \overline{B}$.\end{tabular} \\ \hline
			\begin{tabular}[c]{@{}l@{}}5. Идемпотентность\\ а) $A \vee A \equiv A$,\\ б) $A \wedge A \equiv A$.\end{tabular} &
			\begin{tabular}[c]{@{}l@{}}6. Формулы поглощения\\ а) $A \vee (A \wedge B) \equiv A$,\\ б) $A \wedge (A \vee B) \equiv A$.\end{tabular} \\ \hline
			\begin{tabular}[c]{@{}l@{}}7. Равносильность тождества\\ а) $A \vee Л \equiv A$,\\ б) $A \wedge И \equiv A$.\end{tabular} &
			\begin{tabular}[c]{@{}l@{}}8. Формулы констант\\ а) $A \vee И \equiv И$,\\ б) $A \wedge Л \equiv Л$.\end{tabular} \\ \hline
			\begin{tabular}[c]{@{}l@{}}9. Формулы дополнения\\ а) $A \vee \overline{A} \equiv И$, б) $A \wedge \overline{A} \equiv Л$,\\ в) $\overline{И} \equiv Л$, г) $\overline{Л} \equiv И$.\end{tabular} &
			\begin{tabular}[c]{@{}l@{}}10. Закон инволюции (снятие\\ двойного отрицания)\\ $\overline{\overline{A}} \equiv A$\end{tabular} \\ \hline
			\begin{tabular}[c]{@{}l@{}}11. Формулы расщепления\\ а) $A \equiv (A \wedge B) \vee (A \wedge \overline{B})$,\\ б) $A \equiv (A \vee B) \wedge (A \vee \overline{B})$\end{tabular} &
			\\ \hline
		\end{tabular}
	\end{table}
	
	\newpage
	\begin{center}
		\subsection*{Замена логических операций}
	\end{center}

	\begin{table}[ht]
		\centering
		\begin{tabular}{@{}|l|l|@{}}
			\toprule
			$A \rightarrow B \equiv \overline{A} \vee B$ & $A \rightarrow B \equiv \overline{A \wedge \overline{B}}$  \\ \midrule
			$A \sim B \equiv (A \wedge B) \vee (\overline{A} \wedge \overline{B})$   & $A \sim B \equiv (A \vee \overline{B}) \wedge (\overline{A} \vee B)$   \\ \midrule
			$A \oplus B \equiv (\overline{A} \wedge B) \vee (A \wedge \overline{B})$ & $A \oplus B \equiv (A \vee B) \wedge (\overline{A} \vee \overline{B})$ \\ \midrule
			$A \vee B \equiv \overline{A} \rightarrow B$ & $A \vee B \equiv \overline{(\overline{A} \wedge \overline{B})}$ \\ \midrule
			$A \wedge B \equiv \overline{(A \rightarrow \overline{B})}$         & $A \wedge B \equiv \overline{(\overline{A} \vee \overline{B})}$        \\ \midrule
			$A | B \equiv \overline{(A \wedge B)}$  & $A \downarrow B \equiv \overline{(A \vee B)}$         \\ \bottomrule
		\end{tabular}
	\end{table}
	
	Пусть $X_1, ..., X_n$ - все входящие в формулу A элементарные высказывание. Формула
	\begin{center}
		$A^* (X_1, ..., X_n) \equiv \overline{A} (\overline{X}_1, ..., \overline{X}_n)$
	\end{center}
	называется двойственной к формуле $A$.\\
	\underline{\textit{\textbf{Пример.}}}\\
	1. $A = X \vee Y; A^* (X, Y) \equiv \overline{A} (\overline{X}, \overline{Y}) \equiv \overline{(\overline{X} \vee \overline{Y})} \equiv \overline{\overline{X}} \wedge \overline{\overline{Y}} \equiv X \wedge Y.$\\
	Дизъюнкция $(\vee)$ двойственна конъюнкции $(\wedge)$, и наоборот.
	\\\\
	Пусть формула $A$ зависит от списка переменных $X_1, ..., X_n.$
	\\\\
	Формула $A$ называется \textit{тождественно-истинной} (или \textit{тавтологией}),
	если на всех оценках списка переменных $X_1, ..., X_n$ она принимает значение И.\\
	\underline{\textit{\textbf{Пример.}}} $X \vee \overline{X}.$
	\\\\
	Формула $A$ называется \textit{выполнимой}, если она хотя бы на одной оценке списка
	переменных $X_1, ..., X_n$ она принимает значение И.\\
	\underline{\textit{\textbf{Пример.}}} $\overline{X} \wedge Y.$
	\\\\
	Формула $A$ называется \textit{тождественно-ложной}, если на всех оценках списка
	переменных $X_1, ..., X_n$ она принимает значение Л.\\
	\underline{\textit{\textbf{Пример.}}} $X \wedge \overline{X}.$
	\\\\
	Формула $A$ называется \textit{опровержимой}, если хотя бы на одной оценке списка
	переменных $X_1, ..., X_n$ она принимает значение Л.\\
	\underline{\textit{\textbf{Пример.}}} $X \vee Y.$
	
	\newpage
	\subsection*{Утверждение 1.}
	1. $A$ - тавтология $\Leftrightarrow$ $A$ не является опровержимой.\\
	2. $A$ - тождественно-ложна $\Leftrightarrow$ $A$ не является выполнимой.\\
	3. $A$ - тавтология $\Leftrightarrow$ $\overline{A}$ - тождественно-ложна.\\
	4. $A$ - тождественно-ложна $\Leftrightarrow$ $\overline{A}$ - тавтология.\\
	5. $A \sim B$ - тавтология $\Leftrightarrow$ $A \equiv B.$
	
	\begin{center}
		\subsection*{Некоторые тавтологии (A, B, C - произвольные формулы)}
	\end{center}
	
	\begin{table}[ht]
		\begin{tabular}{@{}|l|l|@{}}
			\toprule
			$A \vee \overline{A}$             & $A \rightarrow A$                            \\ \midrule
			$A \rightarrow (B \rightarrow A)$ & $A \rightarrow (B \rightarrow (A \wedge B))$ \\ \midrule
			$(A \wedge B) \rightarrow A$      & $A \rightarrow (A \vee B)$                   \\ \midrule
			$(A \wedge B) \rightarrow B$      & $B \rightarrow (A \vee B)$                   \\ \midrule
			$(\overline{B} \rightarrow \overline{A}) \rightarrow ((\overline{B} \rightarrow A) \rightarrow B)$ &
			$((A \rightarrow B) \rightarrow A) \rightarrow A$ \\ \midrule
			$(A \rightarrow B) \rightarrow ((B \rightarrow C) \rightarrow (A \rightarrow C))$ &
			$(A \rightarrow (B \rightarrow C)) \rightarrow ((A \rightarrow B) \rightarrow (A \rightarrow C))$ \\ \bottomrule
		\end{tabular}
	\end{table}

	Формулу называют \textit{элементарной дизъюнкцией}, если она является
	дизъюнкцией переменных или отрицаний переменных.\\
	\underline{\textit{\textbf{Пример.}}}\\
	$A_1 (X, Y) = \overline{Y}; A_2 (X, Y, Z) = Z;$\\
	$A_3 (X, Y, Z) = X \vee Y; A_4 (X, Y, Z) = \overline{Z} \vee \overline{Y} \vee X.$
	\\\\
	Формула находится в \textit{конъюнктивной нормальной форме} (КНФ), если она
	является конъюнкцией элементарных дизъюнкций.\\
	\underline{\textit{\textbf{Пример.}}}\\
	$A_1 (X, Y) = Y; A_2 (X, Y) = (Y) \wedge (X);$\\
	$A_3 (X, Y, Z) = (\overline{Y} \vee \overline{X} \vee Z);$\\
	$A_4 (X, Y, Z) = \overline{Z} \wedge (\overline{Y} \vee X);$\\
	$A_5 (X, Y, Z) = (\overline{X} \vee \overline{Z} \vee \overline{Y}) \wedge (X \vee \overline{Y}) \wedge Z.$
	\\\\
	Пусть формула $A$ зависит от $n$ переменных. Формула $A$ находится в \textbf{СДНФ} относительно этих
	переменных, если выполняются следующие условия:\\
	а) $A$ находится в ДНФ (дизъюнкция элементарных конъюнкций);\\
	б) в ней нет двух одинаковых дизъюнктивных членов (т.е. 
	элементарных конъюнкций);\\
	в) каждый дизъюнктивный член (элементарная конъюнкция) формулы $A$
	является $n$-членной конъюнкцией, причем на $i$-ом месте $(1 \leq i \leq n)$
	этой конъюнкции обязательно стоит либо переменная $X_i$, либо ее отрицание
	$\overline{X}_i$.\\
	\newpage
	\underline{\textit{\textbf{Пример.}}} Пусть $(X_1, X_2, X_3)$ - список переменных.\\
	$A (X_1, X_2, X_3) = X_1 \wedge \overline{X}_2 \wedge X_3;$\\
	$B (X_1, X_2, X_3) = (\overline{X}_1 \wedge X_2 \wedge X_3) \vee (X_1 \wedge \overline{X}_2 \wedge \overline{X}_3) \vee (X_1 \wedge X_2 \wedge X_3);$\\
	$C (X_1, X_2, X_3) = (\overline{X}_2 \wedge X_3) \vee (\overline{X}_3 \wedge X_1 \wedge X_2);$
	\\\\
	$A$ и $B$ - СДНФ, $C$ - не является СДНФ.
	\\\\
	Формула $A$ находится в \textbf{СКНФ} относительно списка переменных, если выполняются
	следующие условия:\\
	а) $A$ находится в КНФ (конъюнкция элементарных дизъюнкций);\\
	б) в ней нет двух одинаковых конъюнктивных членов (т.е. элементарных
	дизъюнкций);\\
	в) каждый конъюнктивный член (элементарная дизъюнкция) формулы
	$A$ является $n$-членной дизъюнкцией, причем на $i$-ом месте
	$(1 \leq i \leq n)$ этой дизъюнкции обязательно стоит либо
	переменная $X_i$, либо ее отрицание $\overline{X}_i$.\\
	\underline{\textit{\textbf{Пример.}}} Пусть $(X_1, X_2, X_3)$ - список переменных.\\
	$A (X_1, X_2, X_3) = \overline{X}_1 \vee \overline{X}_2 \vee X_3;$\\
	$B (X_1, X_2, X_3) = (X_1 \vee X_2 \vee X_3) \wedge (X_1 \vee \overline{X}_2 \vee \overline{X}_3) \wedge (X_1 \vee \overline{X}_2 \vee X_3);$\\
	$C (X_1, X_2, X_3) = (\overline{X}_1 \vee X_3 \vee X_2) \wedge (\overline{X}_1);$
	\\\\
	$A$ и $B$ - СКНФ, $C$ - не является СКНФ.
	
\end{document}